%\documentclass[review]{elsarticle}
\documentclass[preprint,3p,twocolumn]{elsarticle}
\usepackage{lineno,hyperref}
\usepackage{color}
\usepackage{amsmath}
\usepackage{subcaption}
\usepackage{verbatim}
\usepackage[percent]{overpic}
\modulolinenumbers[10]

%\journal{NIMA}  %FOR NOW THIS IS AN ARXIV PAPER

%%%%%%%%%%%%%%%%%%%%%%%
%% Elsevier bibliography styles
%%%%%%%%%%%%%%%%%%%%%%%
%% To change the style, put a % in front of the second line of the current style and
%% remove the % from the second line of the style you would like to use.
%%%%%%%%%%%%%%%%%%%%%%%

%% `Elsevier LaTeX' style
\bibliographystyle{elsarticle-num}
%%%%%%%%%%%%%%%%%%%%%%%

\begin{document}

\begin{frontmatter}

\title{DRAFT: Conceptual design of a new neutron detector with neutron localization capabilities}
%\tnotetext[mytitlenote]{Fully documented templates are available in the elsarticle package on \href{http://www.ctan.org/tex-archive/macros/latex/contrib/elsarticle}{CTAN}.}


%% Group authors per affiliation:
\author[mymainaddress]{D. P\'erez-Loureiro}
\ead{dperezlo@utk.edu}
\author[mymainaddress]{J. Heideman}


\author[mymainaddress,ORNLaddress]{R. Grzywacz\corref{mycorrespondingauthor}}
\cortext[mycorrespondingauthor]{Corresponding author}
\ead{rgrzywac@utk.edu}

\author[mymainaddress]{K. Schmitt\fnref{now}}
\fntext[now]{Present address: Los Alamos National Laboratory, Los Alamos, New Mexico 87545, USA}
\author[mymainaddress]{C. R. Thornsberry}
\author[mymainaddress]{S. K. Neupane}

\author[TTUaddress]{M. M. Rajabali}
\author[TTUaddress]{A. R. Engelhardt}
\author[TTUaddress]{C. W. Howell}
\author[TTUaddress]{L. D. Mostella}
\author[TTUaddress]{J. S. Owens}
\author[TTUaddress]{S. C. Shadrick}

\author[JINPAaddress]{S. Munoz}


\author[mymainaddress]{J. Chen}

%\author[otheraddress]{Others}


\address[mymainaddress]{Department of Physics and Astronomy,  University of Tennessee, Knoxville, Tennessee , 37996 USA}
\address[ORNLaddress]{Physics Division, Oak Ridge National Laboratory, Oak Ridge TN 37831 USA}
\address[TTUaddress]{Department of Physics Tennessee Technological University, Cookeville, Tennessee, 38505, USA}
\address[JINPAaddress]{Joint Institute for Nuclear Physics and Applications, Oak Ridge TN 37831 USA}
%\address[otheraddress]{Others' Addresses}


%% abstract
\begin{abstract}
A new high resolution neutron detector concept which time of flight measurements is proposed. Energy resolutions from time of flight measurements are determined by the uncertainty in timing and position. A single NEXT module will be composed of thin segments of neutron-discriminating plastic scintillator, each optically separated, coupled to a position sensitive detector for the readout. The goal of NEXT is to maintain neutron detection efficiency and reduce uncertainties in time of flight energy calculations.
\end{abstract}

\begin{keyword}
$\beta$-delayed neutron emission
\end{keyword}

\end{frontmatter}


\linenumbers

\section{Introduction}
New generation radioactive ion beam facilities enable access to very neutron rich nuclei, approaching, and even reaching the neutron drip-line in certain
cases \cite{FRIB}.
Far from stability, neutron separation energies decrease as beta-decay endpoint energies become large, increasing the likelihood of beta-delayed neutron emission. Neutron spectroscopy becomes essential to obtain important information about the nuclear structure for these very neutron-rich nuclei. 
Neutron dEtector with??(X)?? Tracking (NEXT) has been devoloped to improve neutron energy resolution while still being viable to observe beta-delayed neutron emitters close to the neutron drip line. 
This high-precision neutron detector will also be applicable to single proton transfer reactions used to probe single-particle proton states of exotic nuclei. NEXT has been designed around neutron-gamma discriminating scintillator in order to improve background rejection. Prevelant gamma-ray background common in decay and reaction experimental setups can be removed from neutron spectra, further improving neutron energy measurements \cite{FEBBRARO2018189}.
%\begin{center}
%{ \color{red} TO BE COMPLETED!}
%\end{center}

\section{ Detector Design}

When the neutron kinetic energies are measured via time-of-flight, the energy resolution is given by the following expression \cite{KORNILOV2009226}:
\begin{equation}
\frac{\Delta E}{E}=\sqrt{\left(\frac{2\Delta t}{t}\right )^2+\left(\frac{2\Delta L}{L}\right )^2},
\label{eq:resolution}
\end{equation}
in which $t$  is the time-of-flight of the particle ($\Delta$t is uncertainty in time-of-flight) and $L$ is the corresponding flight path-length ($\Delta$L is the uncertainty in neutron path-length). Therefore, the energy resolution is directly related with the time resolution of the detection system, and the precision in the measurement of the path-length. The latter is mainly due the the uncertainty in the determination of the interaction point within the active volume of the detector. One of the factors which limits this precision is the required thickness facing the trajectory, because the interaction can occur anywhere within the detector. Thick detectors are needed to maintain neutron detection efficiency, but at the expense of a good position resolution. Typical thicknesses for plastic-scintillator-based neutron detectors are about 3~cm, which is a good trade-off between position resolution and efficiency \cite{PETERS2016122,BUTA2000412}.
\begin{figure}[tp]
\centering
%\includegraphics[width=0.33\textwidth]{Figures/NEXT_module.eps}
\includegraphics[width=0.48\textwidth]{Figures/Resolution.eps}
\caption{Energy resolution as a function of the energy of the incoming neutron calculated for different time resolutions and thicknesses for a 1~m distance.}
\label{fig:resolution}
\end{figure}


Most common plastic scintillator exhibits the same response from gamma-ray scattering as neutron scattering, making the two events indiscernible. In order to correct this behavior, many types of neutron-gamma discriminating scintillators have been developed to improve gamma-ray background rejection in neutron time of flight distributions. By utilizing the slower neutron response as compared to gamma-ray scintillation, background can be rejected using varying pulse shape discrimination methods.

\subsubsection{Detector Requirements}

The design of the NEXT module is a compromise between the necessity to obtain highest possible precision in measurement of position and timing of the neutron and the capabilities of the scintillator material and sensor. The optimal, realistic segmentation has to also warrant sufficiently good light collection in order to retain timing and PSD capabilities. Figure \ref{fig:resolution} shows the dependencies of the  neutron energy resolution on the thickness and timing resolution resulting from implementation of Equation \ref{eq:resolution} for a 1~m neutron flight path. We observe that the 3~cm uncertainty in the distance of the existing VANDLE \cite{PETERS2016122} array is the main limitation in the energy resolution of this detector, especially at neutron energies above 2~MeV.

It is apparent that, for a given $\frac{\Delta L}{L}$, the $\frac{\Delta t}{t}$  should be commensurate, meaning that for a particular timing resolution $\Delta$t, and particular L, there is an minimal segment thickness $\frac{\Delta L}{L}$, beyond which the timing resolution dominates the overall energy resolution. For the typical L=50-100~cm and $\Delta$t=1~ns  timing resolution, the segment thickness should be no more than 10~mm. Thin detectors (3-6~mm) will have lower uncertainty in the determination of the position while the contribution due to the time resolution becoming more important.

%\begin{center}
%MERGE WITH ABOVE
%\end{center}
%[The NEXT detector has to be able to accurately measure neutron energies below 10~MeV. This is the region of interest for beta-delayed neutron emission and single nucleon transfer reactions, in which the neutrons emitted at forward angles in the center of mass frame have low energies. Figure \ref{fig:resolution} shows how the energy resolution is  affected by the thickness of the detector and the time resolution for a 1~m flight distance using Eq. \eqref{eq:resolution}. We observe that the 3~cm uncertainty in the distance of the existing VANDLE \cite{PETERS2016122} array is the main limitation in the energy resolution of this detector, especially at neutron energies above 2~MeV. Thin detectors (3-6~mm) will have lower uncertainty in the determination of the position and the contribution due to the time resolution becoming more important.]
\begin{figure}[tp]
  \centering
  \includegraphics[width=\linewidth]{Figures/NExT.pdf}
  \caption{Schematic showing a potential detection configuration for a single NEXT module in a time-of-flight setup. (A) Shown is a top-down view of the detector segmentation along the neutron flight path. (B) shows a side view of the individual segements within the detector, each optically seperated from one another.}
  \label{fig:NEXTschematic}
\end{figure}
NEXT will also have to discriminate neutrons from gamma-rays. Scattered gamma-rays present in the experiment hall can coincide with start signals, causing significant background in the neutron time-of-flight spectrum.
There are several materials for solid scintillators which can provide good neutron-gamma discrimination and good timing, such as stilbene, anthracene, and p-terphenyl. Each of them presently cannot be used as a material for the proposed detector due to limited ability to be machined into segmented array. The recently developed machinable plastic scintilator, Eljen 276 \cite{ZAITSEVA201897}, is a viable base material for the NEXT detector. Developed from Eljen 299, this rigid plastic scintillator can be machined to the desired dimensions of a single detector module. 

\subsection{Detector Concept}
%\begin{figure}[tp]
%\centering
%\includegraphics[width=0.33\textwidth]{Figures/NEXT_module.eps}
%\includegraphics[width=0.25\textwidth]{Figures/NEXT_modules.eps}
%\caption{Conceptual design  drawings of the two possible geometries of a NEXT module detector. Each  module consists of 8 6-mm thick layers of plastic scintillator, 25~cm long}
%\label{fig:NEXT_modules}
%\end{figure}
\begin{figure*}[htp]
 \centering
  \includegraphics[width=0.9\textwidth,trim={3.7cm 0 0 0},clip]{Figures/Geometries.eps}
  \caption{Visualization of a 1 Mev neutron event in NEXT\emph{sim} for two geometries considered for a NEXT layer. Green lines correspond to optical photons produced in the sctintillation.}
  \label{fig:Geometries}
\end{figure*} 


The NEXT array concept is based on multi-layered modules of neutron-gamma discriminating plastic scintillator with a position sensitive photodetectors on both ends of a module. Each module consists of 8 6-mm-thick layers, therefore the total effective thickness is 4.8~cm. Figure \ref{fig:NEXTschematic} shows a possible multi-layered module configuration, with segmentation along the horizontal and vertical directions. The photon sensors considered are either an array of silicon photomultipliers (SiPMs) or flat panel multi anode photomultiplier tubes (MAPMTs). Analysis of the position sensitive photosensor response from detected scintillation light will determine the precise layer in which the neutron scattered, reducing the uncertainy in the neutron flight path-length. The fast timing (sub nanosecond) capabilities of these photosensors will further improve energy resolutions by reducing the time of flight uncertainty. 


\section{Simulations: NEXT\emph{sim}}
\begin{comment}
\begin{figure*}[htp]
 \centering
  \includegraphics[width=0.9\textwidth,trim={3.7cm 0 0 0},clip]{Figures/Geometries.eps}
  \caption{Visualization of a 1 Mev neutron event in NEXT\emph{sim} for two geometries considered for a NEXT layer. Green lines correspond to optical photons produced in the sctintillation.}
  \label{fig:Geometries}
\end{figure*} 
\end{comment}

In order to investigate the light collection efficiency as well as the timing capabilites of the different considered geometries, \emph{NEXTsim}, a {\sc Geant4}-based code was developed \cite{AGOSTINELLI2003250,ALLISON2016186}. This code uses  {\sc Geant4} version 10.1 Patch 3 and outputs to {\sc Root} files for further analysis.
This software simulates the interaction of neutrons, gammas, and charged particles in the matter they traverse. The  physics model (referred as \emph{Physics List} in the {\sc Geant4} context) employed is the recommended QGS\_BERT\_HP, which includes the standard electromagnetic, and the high precision models for neutron scattering, elastic and inelastic, as well as capture and fission.  This model is based on the G4NDL evaluated neutron data library \cite{Apostolakis2009}. In addition to this physics list, the G4OpticalPhysicsList is included to treat the transport of scintillation light as well as the G4RadioactiveDecay to simulate radioactive sources. The generation of the primary particles  was made with the GeneralParticleSource (GPS) module, which allows to define complex beam profiles or source geometries. This module allows to define the primary particle,  spacial and angular distribution, as well as the kinetic energy with simple commands.
 
 Different geometries considered for the NEXT modules were modeled and simulated (see  Fig.~\ref{fig:Geometries}), e.g. rectangular bars and elliptical bars (bars in which the corners were cut in angle to maximize the light focusing in the detector). The user can select between any of the geometries, scintillator, and wrapping materials via macro-driven commands.

For every event, the simulation tracks the incoming particle and registers each interaction in the different layers of matter and stores the information in a \emph{ScintHit} object, which includes position, time and momentum. The collection of hits of each event is stored in a  {\sc Root} \emph{TClonesArray} and then saved in a \emph{Tree}.

%\begin{figure}[bt]
% \centering
% \includegraphics[width=0.48\textwidth]{Figures/LightCollectionEfficiency.eps}
% \caption{Scatter plot of simulated light collection efficiency as a function of the deposited energy by 1 MeV neutrons for the two different geometries shown in Figure \ref{fig:Geometries}.}
% \label{fig:Efficiency}
%\end{figure}

In addition, the code is also able to simulate the scintillation process induced by the scattered particles in it. All the optical photons produced are then tracked until they are detected or absorbed. For each detected photon, their positions and arrival times are recorded in  a \emph{OptPhotonHit} and the hit collection is saved in the same  \emph{Tree} as the emph{ScintHits}.

\subsection{Simulation of light collection efficency}

The different geometries considered for the NEXT detector was simulated in order to determine the light collection efficiency. Figure \ref{fig:Efficiency} shows the results of the light collection efficiency, calculated as the ratio between detected and produced photons, as a function of the energy deposited in the scintillator. In this case the photosensors are considered ideal and every photon hitting the sensitive surface will be detected. We observe that the rectangular geometries have efficiencies close to 50\% on average. The efficiency of the elliptical shape is higher and it reaches up to 68\% due to the focusing effect towards the photosensors.

\subsection{Simulation of photosensor response} \label{sec:photosensorResponse}

The signals produced by the scintillation light in the detectors were  also added  to the simulations. The Single Photo Electron (SPE) response function of the photosensors were modeled and then folded with the optical photon distribution to obtain a realistic photomultiplier signal. The SPE functions for SiPMs and PMTs were  taken from the  Refs. \cite{,Choong2009} respectively. The total response is the sum of the SPEs of each photon arriving at the photo-sensor weighted by the gain of the anode which it struck. The resultant light-response pulse is then given a baseline, electrical noise, and is then ``digitized" by placing it into discrete bins on the y-axis (e.g. from 0 to 65535 to represent a 16-bit digitizer) and discrete time bins on the x-axis (e.g. 4 ns for a 250 MSPS system). The digitized pulses are then integrated to obtain the representative light of the event and are processed with a polynomial constant fraction discrimination algorithm (PolyCFD) \cite{PhDCory} as discussed in section \ref{sec:timingWithSiPMs}. The PolyCFD algorithm computes a time for each pulse which represents the time-of-flight of the incident neutron aggregated from all collected photons. 

\subsection{Simulation of time of flight resolution of the detector}

We have simulated the time of flight spectrum of 1~MeV neutrons impinging onto 10-cm-long plastic scintillator bars in order to determine their timing resolution. Geometrical cross-sections of $6 \times 6$~mm$^{2}$, $3 \times 3$~mm$^{2}$, and $1 \times 1$~mm$^{2}$ were simulated in order to determine the effect of cell width on the measured time resolution. SiPM sensitive surfaces were used for each test. Figure \ref{fig:ToF} shows that an increase in the thickness of the scintillator bar results in an overall broadening of the time of flight resolution.

\begin{figure}[tb]
\centering
\includegraphics[width=0.48\textwidth]{Figures/ToF.eps}
\caption{Distribution of the time of flight for a pencil beam  placed 20 cm away from the detector of 1MeV  neutrons impinging onto different thicknesses scintillators.}
\label{fig:ToF}
\end{figure}

For another test, scintillator plates with cross-sections $6 \times 6$~mm$^{2}$, $3 \times 6$~mm$^{2}$, and $1 \times 6$ mm$^{2}$ and standard $6 \times 6$~mm$^{2}$ SiPM sensitive surfaces on each end were bombarded with a pencil beam of 50,000 $1~MeV$ neutrons traversing a distance of $1~m$. The light response of each plate is computed as in section \ref{sec:photosensorResponse} and the neutron time-of-flight (ToF) is computed as the average of the arrival time of the light pulse for the left and right PMTs. Figure \ref{fig:plateTOF} shows the normalized ToF distributions for each plate overlayed on one another. When a more realistic detector light-response is taken into account, the differences in resolution due to the various widths of the plates are completely washed out by the overall detector time resolution. The FWHM time resolution for the three different plates are all within about $20~ps$ of one another and average to approximately $583~ps$ for the beam of $1~MeV$ neutrons. As expected, the detection efficiency of each of the plates scales linearly with its thickness (i.e. the efficiency of the $6 \times 6$~mm$^2$ plate is two times larger than the $3 \times 6$~mm$^2$ plate which, in turn, is three times larger than the $1 \times 6$~mm$^2$. This means that in addition to posessing twelve times greater efficiency, the $6 \times 6$~mm$^2$ plate exhibits approximately the same detector time resolution as the $1 \times 6$~mm$^2$.  

\begin{figure}[tb]
\centering
\includegraphics[width=0.48\textwidth]{Figures/plateTOF.pdf}
\caption{Distribution of the time of flight for a pencil beam of $1~MeV$ neutrons impinging on scintillator plates $1~m$ away. Normalized ToF distributions are shown for $10~cm$ plates with geometric cross-sections of $6 \times 6$~mm$^{2}$ (blue), $3 \times 6$~mm$^{2}$ (red), and $1 \times 6$~mm$^{2}$ (green).}
\label{fig:plateTOF}
\end{figure}
Based on this result, it was decided that the minimum tile thickness of a prototype should be 6~mm; thinner tiles would not provide any further benefit due to the timing resolution limit of the data acquisition system.

\subsection{Study of neutron backscattering}

Investigating neutron back scattering is very important in a multi-layered detector like NEXT. If a neutron is backscattered to a previous layer this may lead to a wrong determination  of the neutron kinetic energy 

We employed the \emph{NEXTsim} code to evaluate the probability of backscattering in different layers of the detector after 2, 3 and 4 scattering events in the detectors for neutron energies between 100~keV and 5~MeV. The simulation tracks the neutron while traversing the detector layers and we count a backscattering event if the difference between the final and initial layers is negative. Figure  \ref{fig:backscattering} shows the result of the position difference in layers obtained the simulation for a 2~MeV neutron pencil beam. The probability of backscattering is lower than 5\% after a single scatter and after several scatters within the detector volume the probability below 12\%. The effect of backscattering on neutron energy resolutions should therefore be negligible in event reconstruction.

\begin{figure}[tb]
\centering
\includegraphics[width=0.48\textwidth]{Figures/Backscattering2MeV_NEXT.eps}
\caption{Layer difference between the second (blue) and last (red) scattering events within the NEXT detector for a 2 MeV incident neutron.}
\label{fig:backscattering}
\end{figure}

\begin{center}
[Simulation Summary]
\end{center}


%\section{Conclusions}

%%HERE the BIBLIOGRAPHY
%\section*{References}
%%\nocite{*}
%%\bibliography{mybibfile}
%\bibliography{NEXT_NIM}
%
%\end{document}








\section{Detector Prototyping}
The development phase of the NEXT project investigated single segment scintillator prototypes of various geometries and different photosensors. The key point of these tests was to explore whether the scintillation produced in the interaction of neutrons in the plastic would be sufficient enough to retain the timing and neutron-gamma discrimination capabilities under particular detector geometry requirements.

\subsection{Detector and Data Acquisition}
To investigate the timing performance of varying detector setups, coincidence time distributions and time of flight distributions between photosensors attached to opposite ends of plastic scintillators were measured. Two main types of detector setups were tested, the first being small scintillators attached to SiPMs to determine SiPM timing capabilities, and the second was bars of EJ276 coupled to small fast timing PMTs in order to determine the feasability of incorporating n-$\gamma$  discriminating plastic. For both setups, the same data acquisition (DAQ) was used to record signals from the detector setup. The system utilized 16-bit, 250 MHz Pixie-16 digitizers developed by XIA LLC \cite{XIA} to digitize and store traces for later high resolution timing analysis. A detailed description of a similar DAQ setup can be found in \cite{PAULAUSKAS201422}.

\subsection{High Resolution Timing} \label{sec:HRT}
\begin{figure}[bt]
\centering
\includegraphics[width=0.48\textwidth]{Figures/PolyCFD.eps}
\caption{Example of the PolyCFD algorithm on a digitized trace. Green line represents the third order polynomial fit of the maximum. Magenta line shown the linear interpolation, and the blue line the threshold level. The high resolution time is determined by the intersection between the magenta and blue lines and it is represented by the red vertical line.}
\label{fig:PolyCFD}
\end{figure}

\begin{figure}[tp]
\centering
\includegraphics[width=0.5\textwidth, trim={0 0 0 0.5cm}, clip]{Figures/EJ200-SiPMs.eps}
\caption{Top: Two-dimensional histogram of the left-right time difference versus the deposited energy in EJ-200 plastic scintillator from a $^{90}$Sr source, measured with on-board amplification (left) and without it (right). Bottom: Projection on the time axis of the  distributions of top panels. The red line corresponds to the gaussian fit used to determine the time resolution.}
\label{fig:SiPMtiming}
\end{figure}
Time-of-flight as well as position of the scintillation interaction between two photo-detectors can be determined using the time differences between the signals from each end of a length of scintillator. The internal timestamping of the XIA Pixie-16 digitizers is only in 8~ns intervals so a method to determine a more precise timing was implemented.
The high resolution time of each digitized pulse was determined by means of a Polynomial Constant Fraction Discrimination  algorithm (PolyCFD) \cite{PhDCory}. The algorithm calculates the maximum from a polynomial fit around the peak of the digitized pulse and it calculates the CFD threshold as a fraction of the difference between the maximum and the baseline. The high resolution timestamp is found from a linear interpolation between the points surrounding the CFD threshold in the leading edge. The optimum threshold fraction values were obtained for a factor range between $F=40-45\%$. A graphical representation of the PolyCFD method can be seen in Figure \ref{fig:PolyCFD}.

A $^{90}$Sr source was employed to measure left-right position resolution because of the effective collimation due to the short range of the beta particles within the scintillator material. The $^{90}$Sr source also provided a wide range of energy depositions from tens of keV up to $\sim$2~MeV. This helped establish the timing performance of different detector setups as a function of energy deposition in the scintillator. To determine time-of-flight capabalities of varying detector setups, neutrons from a readily available $^{252}$Cf fission source were observed and their respective flight times were measured between START and STOP detectors. The $^{252}$Cf neutron emission is very well characterized and is provides a good test of a time-of-flight detector's capabilities. 

\section{Timing with SiPMs} \label{sec:timingWithSiPMs}

Silicon Photomultipliers (SiPMs) offer a small form factor solution to detector design and quantum efficiency uniformity for multi-detector arrays. SiPM position and time-of-flight resolution was measured to test their applicability to a small scale array. Two different readout circuits were designed for the SiPM timing measurements to measure the effects of the onboard filtering/amplification. The first one consisted of a simple low-pass active filter based on the Texas Instruments\textsuperscript{\textregistered} OPA656 operational amplifier recommended by the SiPM manufacturer \cite{JseriesUM}. The feedback resistor value was chosen to be $25\Omega$ to maintain the fast rise-time of the SiPM signal while filtering high-frequency noise. The second circuit tested does not have any on-board amplification. Pairs of the same SiPM signal readout boards were tested with a teflon-wrapped $50\times6\times6$~mm$^3$ piece of EJ-200 plastic scintillator between them. For each pair, the signals were amplified and gain matched using an ORTEC\textsuperscript{\textregistered} 535 fast amplifier module.

\subsection{SiPM Position Resolution}
Figure \ref{fig:SiPMtiming} shows the results of the timing measurements using the teflon-wrapped piece of EJ-200 with the two circuits mentioned previously. The left panels correspond to the circuit with active onboard amplification and right ones the the circuit with no onboard amplification. The top panels show 2D-histograms of the left-right time difference and the deposited energy in the plastic scintillator with on-board SiPM amplification (left) and without amplification (right). The maximum of the energy distribution in the case of the SiPM circuit with amplification is lower than the one without it. This is due to the fact that signals with amplitudes larger than 1~V saturate the ADCs and were not included in the histogram. The gain of the amplifier included in the SiPM circuit boards produces the saturation at smaller energy deposition. It is also worth mentioning that the the distribution corresponding to the non amplified circuit shows a significant walk effect at high energy deposition compared to the amplified board.
\begin{figure}[t]
  \includegraphics[width=\linewidth]{Figures/ToF_Cf252_Ej200SiPM.eps}
  \caption{Two-dimensional histogram of $^{252}$Cf time of flight measurements plotted against deposited energy in the EJ-200 stop detector. The inset show a logarithm projection onto the time of flight axis. The timing resolution was determined to be 544~ps from a gaussian fit to the gamma peak.}
  \label{fig:ToF_SiPM}
\end{figure}

Lower panels of Figure \ref{fig:SiPMtiming}, show projections on the time axis of the histograms shown in the upper panel. The time resolution is obtained from  gaussian fits of the time distributions. The resolutions obtained for the amplified circuit and the non-amplified one are $\Delta t=469$~ps (FWHM), and $\Delta t=548$~ps (FWHM) respectively.

\subsection{Time of Flight with SiPMs}

As a proof of principle, a small scale ToF setup (30~cm flight path) was implemented to measure neutron flight times from a $^{252}$Cf source. The setup consisted of a $20\times6\times6$~mm$^3$  piece of EJ-200 plastic scintillator attached to $6\times6$~mm$^2$ SensL\textsuperscript{\textregistered} SiPMs used as a START detector and a  $100\times6\times6$~mm$^3$ scintillator bar attached to another pair of SensL\textsuperscript{\textregistered} SiPMs used as the STOP detector. Both detectors were placed 30~cm apart and the same DAQ was used from the position resolution tests. A ToF vs. deposited energy distribution can be seen in Figure \ref{fig:ToF_SiPM}, along with a 1-D projection showing the gaussian fit to the gamma-ray peak. Using the PolyCFD method, the time of flight resolution for the SiPM setup was determined to be $\Delta t=544$~ps (FWHM).
\begin{figure}[bt]
  \centering
  \begin{subfigure}{0.5\linewidth}
    \raggedleft
    \includegraphics[scale=0.2]{Figures/EJ276_SingleMylarBar_smallPMT_Sr90_LRvsEdep.eps}
  \end{subfigure}%
  \begin{subfigure}{0.5\linewidth}
    \raggedright
    \includegraphics[scale=0.18]{Figures/EJ276_SingleMylarBar_smallPMT_Sr90_LR.eps}
  \end{subfigure}%
  \caption{Left: Two-dimensional histogram of the time difference between PMTs on opposite ends of a 254~mm ESR wrapped bar of EJ-276. Right: Projection of 2D histogram on the time axis. A gaussian fit to the distribution is shown in red.}
  \label{fig:MylarTiming}
\end{figure}

The small scale SiPM timing tests establish SiPMs as viable detectors for small scale arrays. Testing will continue to determine scalability of SiPMs to a large, multi-detector resistive readout system.

\section{Eljen 276 Detector Tests}

 \subsection{Timing Tests with PMTs}
\begin{figure}[tp]
  \centering
 \begin{overpic}[scale=.4]{Figures/EJ276SingleBar_ToFvsEdep_Cf252.eps}
 \put(40,25){\includegraphics[scale=.224, trim={1cm 0 1.5cm 1cm}, clip]{Figures/EJ276SingleBar_ToF_Cf252.eps}}
 \end{overpic}
 \caption{Two-dimensional histogram of $^{252}$Cf time of flight versus deposited energy in the mylar wrapped EJ-276 stop detector. The inset is a projection of the gamma-ray peak in the time of flight spectrum and has a 538~ps FWHM [50 keVee threshold].}
 \label{fig:TOFEJ276}
\end{figure}
 
Timing performance of $127\times12.7\times6$ mm\textsuperscript{3} EJ-276 bars was tested using fast, compact Hamamatsu R11265U photomultipliers. The bars of EJ-276 were machined from 12.7~mm thick sheets by Agile Technologies, Inc. and were wrapped with either 3M\textsuperscript{\texttrademark} ESR (Enhanced Specular Reflector) or Lumirror\textsuperscript{\texttrademark} (produced by Toray). Some bars were also provided with no wrapping to determine the effect of the reflective layers.
\begin{figure*}[tbp]
 \centering
  \includegraphics[width=0.9\textwidth]{Figures/PSDWrappingComparison.eps}
  \caption{Two dimensional histograms of the CCM PSD for three different types of wrapping.}
  \label{fig:PSDEJ276}
\end{figure*}
ESR is a specular reflector with 98\% reflectivity in the visible spectrum and Lumirror\textsuperscript{\texttrademark} is a diffuse reflector similar to teflon. Both wrappings were applied to the EJ-276 bars using an UV-cured optical adhesive. Figure \ref{fig:MylarTiming} shows the 2D-histogram of left-right time difference plotted against the deposited energy in the ESR-wrapped EJ-276 scintillator. The inset in Figure \ref{fig:MylarTiming} shows the y-axis projection of the timing resolution as a gaussian distribution. From a fit to the time difference projection, the left-right timing resolution is $\Delta t=543$~ps (FWHM). A thick mylar film on the $^{90}$Sr source caused the energy spectrum to be compressed by approximately 500~keV, as seen in the endpoint energy rougly around 1700~keV in Fig. \ref{fig:MylarTiming}.  The Lumirror\textsuperscript{\texttrademark} wrapped detector was not tested for timing due to the poor neutron-gamma discrimination that will explained in Section \ref{sec:PSD}.
A ToF setup similar to the SiPM test was made to measure the time of flight resolution for a single 127~mm long bar of EJ-276 plastic scintillator wrapped with ESR. Utilizing the same start detctor from the SiPM time of flight setup and the EJ-276 bar coupled to PMTS as the stop detector, the $^{252}$Cf ToF spectrum was measured. From a fit to the gamma-ray peak in the the time of flight spectrum, the time resolution was determined to be $\Delta t=538$~ps [50keVee threshold applied].

\subsection{neutron-${\mathit \gamma}$ discrimination} \label{sec:PSD}
EJ-276 evolved from first-generation neutron-gamma discriminating plastic scintillator, EJ-299. Typical solid-state scintillators are cast or grown in specific molds, limiting detector designs. EJ-276 is capable of being cut and polished in desired geometries to optimize detector light collection. The pulse shape discrimination (PSD) response of EJ-276 is the same as EJ-299 with a slower neutron decay component than the gamma-ray response. The PSD response mechanism for EJ-299 can be found accurately described in \cite{Zaitseva2012}. The materials response to neutron and gamma-ray scattering was tested for the long narrow segments with different wrappings.

\begin{figure}[t]
  \centering
 \begin{subfigure}{0.5\textwidth}
  \centering
  \includegraphics[width=0.65\linewidth]{Figures/PrototypeSegmentation_axis.png}
 \end{subfigure}% 
  \\
\begin{subfigure}{0.5\textwidth}
  \centering
  \includegraphics[width=0.75\linewidth]{Figures/PrototypePositionMap.eps}
\end{subfigure}%
\caption{The top figure is an image of one end of a 4x8 segmented scintillator. The bottom figure shows the reconstructed cells using the position sensitive signals from the Vertilon Interface board. The detector is always arranged such that the 4 rows are parallel to the particle flight path and the 8 columns are perpendicular. The higher segmentation is along the flight path for the best position and timing resolution.}
  \label{fig:PSPMTImage}
\end{figure}
%\subsubsection{Wrapping Tests}
 Two wrapped bars from Agile Technologies, Inc. (3M\textsuperscript{\texttrademark} ESR and Lumirror\textsuperscript{\texttrademark}) and a third bare bar wrapped with teflon were tested to measure the effect of the outer reflective layer on the pulse shape discrimination. The neutron and gamma-ray responses from EJ-276 were recorded with the same 16 bit 250 MHz digitizer used for earlier setups, and the pulse shape discrimination was tested using the Charge Comparison Method \cite{CCMPSD}. By measuring the total and partial (tail) integral of each signal and calculating the ratio between the two integrals, neutrons scattering events can be separated from gamma-ray scatters. This approach is further improved by utilizing both ratios from detectors on either end of a scintillator bar and calculating the geometric mean. This method has previously been determined to be optimal when using a high bit-resolution digitizer \cite{HighResPSD}.
Using the $^{252}$Cf source and a 2 inch block of lead to attenuate the large gamma-ray flux, the waveforms from each bar were digitized and tail to total integral ratios were calculated. Figure \ref{fig:PSDEJ276} shows the PSD plots for bars wrapped with Lumirror\textsuperscript{\texttrademark} (a), teflon (b), and ESR (c) with figure of merits being calculated between 400 and 500~keVee. The figure of merits for the Lumirror\textsuperscript{\texttrademark}, teflon, and ESR bars were 0.820$\pm$0.012, 1.042$\pm$0.016, and 0.977$\pm$0.015 respectively. Further tests of the bar were only done with 3M\textsuperscript{\texttrademark} ESR wrapping to maintain the best neutron-gamma discrimination.

\section{NEXT Prototype} \label{Prototype}

ESR wrapped individual EJ-276 bars were shown to meet NEXT design goals, leading to the assembly of 2x2 in$^{2}$ segmented detectors. The prototype has 4x8 scintillator cells, the higher segmentation being along the particle flight path (see Figure \ref{fig:PSPMTImage}). A full NEXT prototype is made up of one segmented array coupled to Hamamatsu H12700A position sensitive PMTs (PSPMTs) on each end of the segmented scintillator. The H12700A PSPMTs have an 8x8 segmentation (64 6x6 mm$^{2}$ anodes), each anode having an individual readout. A Vertilon PSPMT Anger Logic interface board (Model SIB064B-1018) was used to reduce the position sensitive readout from 64 individual position signals to 4 position signals, one at each corner of the SIB064B-1018 resistive network. The scintillation position is reconstructed using the weighted average of the 4 corner resistive network signals based on their respective signal integrals \cite{ANGER}. Figure \ref{fig:PSPMTImage} shows the reconstructed array segmentations from the Anger Logic position measurement of the PSPMTs using scintillation data from a $^{60}$Co source. The PSPMT common dynode signal used for timing is connected directly to the acquistion. The scintillator cell dependent analysis calculates neutron energies on a segment by segment basis (using reconstructed positions).

\subsection{Time of Flight Measurements}
\begin{figure}[tp]
 \centering
% \includegraphics[width=0.9\linewidth]{Figures/CollimatedCo60_PrototypeToF.eps}
 \includegraphics[width=0.9\linewidth,trim={0 0 0 1cm},clip]{Figures/CollimatedCo60_PrototypeToF_MiddleSegment.eps}
 \caption{Time of flight resolution or a single NEXT prototype column using a collimated $^{60}$Co source at a flight distance of 439.7~mm. The gaussian fit to the distribution shows the time resolution (FWHM) is 543~ps. [30 keVee threshold]}
 \label{fig:CollimatedCoToF}
\end{figure}
\begin{figure}[tb]
  \centering
  \includegraphics[width=0.9\linewidth]{Figures/Cf252_Spectrum_logY.eps}
  \caption{$^{252}$Cf neutron energy spectrum as measured with the NEXT prototype using the segment dependent analysis (blue). The red line shows the expected neutron yield based on a 100 keVee detection threshold.}
  \label{fig:Cf252Spectrum}
\end{figure}


\begin{figure}[t]
  \centering
 \begin{overpic}[scale=.35]{Figures/Prototype_PSD.eps}
 \put(35,35){\includegraphics[scale=.19]{Figures/Prototype_PSD_FOM.eps}}
 \end{overpic}
 \caption{Neutron-gamma discrimination from the common dynode signal of the PSPMT. The walk in the distribution has been removed by correcting the gamma-ray portion to a flat line on a bin by bin basis.}
 \label{fig:PSPMTPSD}
\end{figure}

To measure the prototype time of flight resolution, a collimated $^{60}$Co source was used. Figure \ref{fig:CollimatedCoToF} shows the time of flight distribution for a single column. From a gaussian fit to the distribution, the time of flight resolution is $\Delta$t=543~ps [30 keVee threshold]. Once the prototype was established to have high resolution timing, a proof-of-principle neutron energy measurement was made using the $^{252}$Cf source. The source was placed at a distance of 439.7~mm from the front face of the prototype. The neutron yield shown in Figure \ref{fig:Cf252Spectrum} was calculated using time-of-flight information and a neutron gate has been applied using the PSD information shown in Figure \ref{fig:PSPMTPSD}. The disagreement at low neutron energy is likely due to a stringent detection threshold in simulated efficiency data which was folded with the Watt equation. The standard Mannhart parameters were fixed for the Watt equation in a fit to the spectrum, with an additional scaling parameter which was floated to match the data \cite{Mannhart}. 

\subsection{neutron-${\mathit \gamma}$ discrimination}
The PSPMT response is different then that of the fast timing PMTs used to initially test EJ-276. This response affects the overall pulse shape, potentially affecting the pulse shape discrimination capabilities. The dynode signals contain the pulse shape discrimination information. The four position signals lose neutron-gamma information after passing through the resistive network. Figure \ref{fig:PSPMTPSD} shows the neutron-gamma discrimination using the CCM. Using the same energy cuts from the wrapping tests, the FOM is 1.070$\pm$0.016. The NEXT protype does not show any noticeable effect on neutron-gamma discrimination due to segmentation or multi-anode readout.

\begin{comment}
\begin{figure*}[ht]
  \centering
  \includegraphics[width=0.8\textwidth]{Figures/UKsetup.png}
  \caption{NEXT UKAL setup for measuring mono-energetic neutrons. Show here is the collimator and detector at 95 degrees with respect to initial proton direction, which correlates to detecting 526 keV neutrons. The DAQ is mounted behind the detector encapsulating the power supply, external amplification and digitizers.}
  \label{fig:UKsetup}
\end{figure*}
\end{comment}

\section{Mono-Energetic Neutron Tests}
NEXT's defining charactersitic is high precision, position dependent timing correlation. When neutrons pass through the segmented detector, there is a non-negligible amount of time a neutron takes to traverse the thickness of a single column. Neutron ToF measurements should therefore correspond to where within the detector the neutron interacted, e.g. neutrons that are determined to have interacted with a deeper column should have longer time-of-flights compared to those which interact in the front segments. For mono-energetic neutrons detected with the segmented NEXT prototype, the ToF distribution for each successive column should shift by the time it takes a neutron to traverse a single column thickness. In order to benchmark the timing-position correlation for the NEXT prototype, mono-energetic neutrons were measured at the Univesity of Kentucky Accelerator Laboratory (UKAL).
\begin{figure}[t]
  \includegraphics[width=\linewidth, trim={0 0.2cm 0 1.1cm}, clip]{Figures/Sim_1013keV_ToFvsSeg.pdf}
  \caption{ToF means for each segment from simulated 1013 keV neutrons detected by the NEXT prototype using the NEXT\emph{sim} framework. The dashed red line shows the expected position dependence of the time of flight measurements for 1013 keV neutrons.}
  \label{fig:simToFvsSeg}
\end{figure}
\begin{figure}[t]
  \centering
  \includegraphics[width=\linewidth, trim={0 0.2cm 0 1.1cm}, clip]{Figures/Official_ToFvsSeg_3prototypes_1013keV_noFits.pdf}
  \caption{Plots showing the ToF shift per segment for each prototype: EJ276-10 (green), EJ276-05 (red), and EJ200-10 (blue). Data shown corresponds to time-of-flight measurements of $\sim$1~MeV neutrons. The data has been shifted appropriately to lie on the same scale and the associated errors are based solely on the statistical uncertainty in the determination of the mean. The black dashed line is the expected $\Delta ToF$ based on neutron ToF calculations.}
  \label{fig:ToFvsSeg}
\end{figure}
\subsection{Experimental Setup}
At UKAL, neutrons are generated in a t(p,n)$^{3}$He reaction, a d(d,n)$^{3}$He reaction, or a t(d,n)$^4$He reaction. An in-depth overview which describes the neutron production and energy selection of the UKAL can be found in \cite{HARTMAN2015137}. The t(p,n)$^{3}$He reaction was used to generate neutron energies in the 0.248 to 1.5 MeV range, but only $\sim$1~MeV neutrons will be discussed below as an example of NEXT's position dependent timing characteristics. NEXT was situated behind stacked copper, polyethylene/lead, and paraffin/lithium carbonate collimators and aligned at 55\textdegree~w.r.t. the proton beam direction, corresponding to 1013 keV neutrons. Due to the acceptance window of the detctor and collimators, observed neutron energies ranged from [??APPROX.??] 993 keV to 1035 keV.  Using the same 16 bit, 250 MHz pixie-16 based acquisition system as with previous setups, neutron time-of-flights were measured as the time difference between the proton beam pickoff signal before the tritium target and the dynode timing signals from the NEXT prototype. The specialized XIA LLC\textsuperscript{\textregistered} pixie-16 firmware allowed the acquisition to be run in triple coincidence \cite{PAULAUSKAS201422}. Although only one prototype was tested at a time, each detector requires 10 pixie channels, which can saturate the digitizer with a much lower count rate. The triple coincidence firmware has internal triggering that will only write a full event if the one start signal (proton beam pickoff) and the two stop signals (left-right dynode timing signals) both trigger within the pre-determined coincidence window. This helped lower triggering thresholds into electrical noise levels while still maintaining a stable, unsaturated count rate.

\subsection{Simulating Time-of-Flight Propogation}
\begin{comment}
\begin{table}[t]
\caption{Slopes in $\left[\sfrac{ns}{col}\right]$ from three different ranged fits to the data shown in Figure \ref{fig:simToFvsSeg}.}
\label{tab:simSlopes}
\begin{center}
\begin{tabular}{c c c}
 \hline
  Col. 1-8 & Col. 1-7 & Col. 2-7 \\
 \hline
 \hline
  $0.456\pm0.020$  & $0.486\pm0.022$ & $0.453\pm0.027$ \\
 \hline
\end{tabular}
\end{center}
\end{table}
\end{comment}
[EXPLAIN SIMULATION BEHAVIOR]
 Ideally, the shift in the mean of ToF distributions for each successive segment would be constant. A simulation replicating UKAL NEXT measurements was done to provide an estimate of the detector response to $\sim$1 MeV neutrons. 1013 keV neutrons in a pencil beam with beamspot radius 25.4~mm were simulated along a 3.1~m flight path to the front of a NEXT prototype. Only 1013~keV neutrons were simulated because the neutron energy distribution of the UKAL setup has not been fully studied. Using the full NEXT\emph{sim} simulation (GEANT4 interactions and photosensor response), neutron ToF were measured and the mean of each segments distribution was plotted against the corresponding segment number. In order to compute the optical photon center-of-mass (segment position), the weighted average is taken of the X and Y positions of each individual photon detected at the surface of the PSPMT. Each detection event is weighted using the product of the gain of the anode which was hit and the quantum efficiency of the PSPMT for a given wavelength. In doing so, a position map similar to what is shown in Figure \ref{fig:PSPMTImage} can be made using simulated data.
Figure \ref{fig:simToFvsSeg} shows the expected prototype position dependent timing behavior when detecting $\sim$1~MeV neutrons. The dashed red line is the expected fit based on 1013 keV neutron ToF calculations. The same methods used to cut on the resistive network event position in experimental data were used for position selections from optical photon center of mass calculations in the simulated data. 1013 keV neutrons traverse a single cell thickness (6.35~mm) in 0.457~ns. The first row in Table \ref{tab:Slopes} shows the shift in ToF per column ($\Delta ToF$) determined by fitting a first order polynomial to the simulated data. From the fit, $\Delta ToF$ is 0.456~ns.  

\subsection{Preliminary Results}
\begin{table}[t]
\caption{Slopes $\Delta ToF$ in $\left[\sfrac{ns}{col}\right]$ from first order polynomial fit to simulated and real ToF data for 1013~keV neutrons.}
\label{tab:Slopes}
\begin{center}
\begin{tabular}{p{0.2\linewidth} c}
 \hline
 Prototype & $\Delta ToF$ $\left[\frac{ns}{col}\right]$  \\
 \hline
 \hline
 NEXT\emph{sim}    & $0.456\pm0.020$  \\
 EJ276-10  & $0.439\pm0.013$  \\
 EJ276-05  & $0.402\pm0.013$  \\
 EJ200-10  & $0.424\pm0.010$  \\
 \hline
\end{tabular}
\end{center}
\end{table}
%\begin{table}[ht]
%\caption{Slopes [$\frac{\Delta ToF}{col}$] from first order polynomial fit to time-of-flight data for 1013 keV neutrons}
%\label{tab:Slopes}
%\begin{center}
%\begin{tabular}{p{0.1\linewidth} c c c}
% \hline
% Prototype &    EJ276-10     &     EJ276-05     &    EJ200-10 \\
% \hline
% \hline
% Fit [1-8] & $0.439\pm0.013$ & $0.402\pm0.013$  & $0.424\pm0.010$  \\
% Fit [1-7] & $0.473\pm0.015$ & $0.421\pm0.014$  & $0.440\pm0.011$  \\
% \hline
%\end{tabular}
%\end{center}
%\end{table}
To demonstrate the feasability of NEXT, a preliminary analysis was done on $\sim$1~MeV neutrons detected in three different segmented NEXT prototypes: EJ276-10 (10 inch EJ-276 4x8 array), EJ276-05 (5 inch EJ-276 4x8 srray) and EJ200-10 (10 inch EJ-200 4x8 array). For 1013 keV neutrons, the mean of the ToF distributions should shift by 0.432~ns for each successive column. The cell thickness is slightly less than used in the simulation due to manufacturing constraints, so the expected $\Delta$ is $\sim$20~ps less than was was calculated for the simulated data (0.457~ns). By making cuts for each column, the mean of each column ToF distribution was extracted using a gausian fit to 1.5$\sigma$ around the maximum. $\Delta ToF$ for each prototype was determined from a first order polynomial fit to the mean time-of-flight vs. column data shown in Figure \ref{fig:ToFvsSeg}.
\begin{comment}
  The last column in each prototype had significantly less shift than all other columns, so a fit was also performed to columns 1 to 7 as well to observe the effect of column 8 on the overall time-of-flight propagation.
\end{comment}
The $\Delta ToF$ values for the fit to each prototypes are shown in Table \ref{tab:Slopes}. Overall, each detector showed expected position dependent timing characteristics, with a clear shift in ToF measurements from column to column. The largest difference from the expected slope was observed in the EJ276-05 prototype.
Column 8 consistently had a smaller shift in ToF possibly due to improper position selection due to low statistics for this column or as a physical effect of the detector.
Continuing analysis on this data set will be performed along with other ToF data acquired for different neutron energies.



\section{Conclusions}
After extensive development guided by simulations and single segment tests, a NEXT prototype has been built with 4x8 segmentation. The SensL\textsuperscript{\texttrademark} J-Series SiPMs successfully measured neutron time of flight with less than 600 ps timing resolution, validating SiPMs as potential detectors for future development.

%HERE the BIBLIOGRAPHY
\section*{References}
%\nocite{*}
%\bibliography{mybibfile}
\bibliography{NEXT_NIM}

\end{document}
