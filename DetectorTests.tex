%\documentclass[review]{elsarticle}
%\documentclass[preprint,3p,twocolumn]{elsarticle}
%\usepackage{lineno,hyperref}
%\usepackage{color}
%\usepackage{amsmath}
%\usepackage{subcaption}
%\usepackage{verbatim}
%\usepackage[percent]{overpic}
%\modulolinenumbers[10]
%
%\journal{NIMA}
%
%%%%%%%%%%%%%%%%%%%%%%%%
%%% Elsevier bibliography styles
%%%%%%%%%%%%%%%%%%%%%%%%
%%% To change the style, put a % in front of the second line of the current style and
%%% remove the % from the second line of the style you would like to use.
%%%%%%%%%%%%%%%%%%%%%%%%
%
%%% `Elsevier LaTeX' style
%\bibliographystyle{elsarticle-num}
%%%%%%%%%%%%%%%%%%%%%%%%
%
%\begin{document}
%
%\begin{frontmatter}
%
%\title{DRAFT: Conceptual design of a new neutron detector with neutron localization capabilities}
%%\tnotetext[mytitlenote]{Fully documented templates are available in the elsarticle package on \href{http://www.ctan.org/tex-archive/macros/latex/contrib/elsarticle}{CTAN}.}
%
%
%%% Group authors per affiliation:
%\author[mymainaddress]{D. P\'erez-Loureiro}
%\ead{dperezlo@utk.edu}
%\author[mymainaddress]{J. Heideman}
%
%
%\author[mymainaddress,ORNLaddress]{R. Grzywacz\corref{mycorrespondingauthor}}
%\cortext[mycorrespondingauthor]{Corresponding author}
%\ead{rgrzywac@utk.edu}
%
%\author[mymainaddress]{K. Schmitt\fnref{now}}
%\fntext[now]{Present address: Los Alamos National Laboratory, Los Alamos, New Mexico 87545, USA}
%\author[mymainaddress]{C. R. Thornsberry}
%\author[mymainaddress]{S. K. Neupane}
%
%\author[TTUaddress]{M. M. Rajabali}
%\author[TTUaddress]{A. R. Engelhardt}
%\author[TTUaddress]{C. W. Howell}
%\author[TTUaddress]{L. D. Mostella}
%\author[TTUaddress]{J. S. Owens}
%\author[TTUaddress]{S. C. Shadrick}
%
%\author[JINPAaddress]{S. Munoz}
%
%
%\author[mymainaddress]{J. Chen}
%
%%\author[otheraddress]{Others}
%
%
%\address[mymainaddress]{Department of Physics and Astronomy,  University of Tennessee, Knoxville, Tennessee , 37996 USA}
%\address[ORNLaddress]{Physics Division, Oak Ridge National Laboratory, Oak Ridge TN 37831 USA}
%\address[TTUaddress]{Department of Physics Tennessee Technological University, Cookeville, Tennessee, 38505, USA}
%\address[JINPAaddress]{Joint Institute for Nuclear Physics and Applications, Oak Ridge TN 37831 USA}
%%\address[otheraddress]{Others' Addresses}
%


\section{Detector Prototyping}
The development phase of the NEXT project investigated single segment scintillator prototypes of various geometries and different photosensors. The key point of these tests was to explore whether the scintillation produced in the interaction of neutrons in the plastic would be sufficient enough to retain the timing and neutron-gamma discrimination capabilities under particular detector geometry requirements.

To investigate the timing performance of varying detector setups, coincidence time distributions and time of flight distributions between photosensors attached to opposite ends of plastic scintillators were measured. Two main types of detector setups were tested, the first being small scintillators attached to SiPMs to determine SiPM timing capabilities, and the second was bars of EJ276 coupled to small fast timing PMTs in order to determine the feasability of incorporating n-$\gamma$  discriminating plastic. The pulse shape disrminiation of EJ-276 was also tested with the same photosensors For both setups, the same data acquisition (DAQ) was used to record signals from the detector setup. The system utilized 16-bit, 250 MHz Pixie-16 digitizers developed by XIA LLC \cite{XIA} to digitize and store traces for later high resolution timing analysis. A detailed description of a similar DAQ setup can be found in \cite{PAULAUSKAS201422}.

\subsection{High Resolution Timing}
\begin{figure}[bt]
\centering
\includegraphics[width=0.48\textwidth]{Figures/PolyCFD.eps}
\caption{Example of the PolyCFD algorithm on a digitized trace. Green line represents the third order polynomial fit of the maximum. Magenta line shown the linear interpolation, and the blue line the threshold level. The high resolution time is determined by the intersection between the magenta and blue lines and it is represented by the red vertical line.}
\label{fig:PolyCFD}
\end{figure}

\begin{figure}[tp]
\centering
\includegraphics[width=0.5\textwidth]{Figures/EJ200-SiPMs.eps}
\caption{Top: Two-dimensional histogram of the left-right time difference versus the deposited energy in EJ-200 plastic scintillator from a $^{90}$Sr source, measured with on-board amplification (left) and without it (right). Bottom: Projection on the time axis of the  distributions of top panels. The red line corresponds to the gaussian fit used to determine the time resolution.}
\label{fig:SiPMtiming}
\end{figure}
Time-of-flight as well as position of the scintillation interaction within a detector can be determined using the time differences between the signals from each detector attached to the ends of plastic scintillator. The internal timestamping of the XIA Pixie-16 digitizers is only in 8~ns intervals so a method to determine a more precise timing was implemented.
The high resolution time of each digitized pulse was determined by means of a Polynomial Constant Fraction Discrimination  algorithm (PolyCFD) \cite{PhDCory}. The algorithm calculates the maximum from a polynomial fit around the peak of the digitized pulse and it calculates the CFD threshold as a fraction of the difference between the maximum and the baseline. The high resolution timestamp is found from a linear interpolation between the points surrounding the CFD threshold in the leading edge. The optimum threshold fraction values were obtained for a factor range between $F=40-45\%$. A graphical representation of the PolyCFD method can be seen in Figure \ref{fig:PolyCFD}.

A $^{90}$Sr source was employed to measure left-right position resolution because of the effective collimation due to the short range of the beta particles within the scintillator material. The $^{90}$Sr source also provided a wide range of energy depositions from tens of keV up to ~2~MeV. This helped establish the timing performance of different detector setups as a function of energy deposition in the scintillator. To determine time-of-flight capabalities of varying detector setups, neutron time-of-flight was measured using a $^{252}$Cf fission source. The $^{252}$Cf neutron emission is very well characterized and is provides a good test of a time-of-flight detector capabilities. 

\section{Timing with SiPMs} \label{sec:timingWithSiPMs}

Silicon Photomultipliers (SiPMs) offer a small form factor solution to detector design and quantum efficiency uniformity for multi-detector arrays. SiPM position and time-of-flight resolution was measured to test their applicability to a small scale array. Two different readout circuits were designed for the SiPM timing measurements, one with on-board filtering and amplification and another one without them.  The first one consisted of a simple low-pass active filter based on the Texas Instruments\textsuperscript{\textregistered} OPA656 operational amplifier recommended by the SiPM manufacturer. The feedback resistor value was chosen to be $25\Omega$ to maintain the fast rise-time of the SiPM signal while filtering high-frequency noise. The second circuit tested does not have any on-board amplification. Pairs of the same SiPM signal readout boards were tested with a teflon-wrapped $50\times6\times6$~mm$^3$ piece of EJ-200 plastic scintillator between them. For each pair, the signals were amplified and gain matched using an ORTEC\textsuperscript{\textregistered} 535 fast amplifier module.

Figure \ref{fig:SiPMtiming} shows the results of the timing measurements using the teflon-wrapped piece of EJ-200 with the two circuits mentioned previously. The left panels correspond to the circuit with active onboard amplification and right ones the the circuit with no onboard amplification. The top panels show 2D-histograms of the left-right time difference and the deposited energy in the plastic scintillator with on-board SiPM amplification (left) and without amplification (right). We observe that the maximum of the energy distribution in the case of the SiPM circuit with amplification is lower than the one without it. This is due to the fact that signals with amplitudes larger than 1~V saturate the ADCs and were not included in the histogram. The gain of the amplifier included in the SiPM circuit boards produces the saturation at lower values of deposited energy. It is also worth to mention that the the distribution corresponding to the non amplified circuit shows a significant walk effect at high energy deposition compared to the amplified board.
\begin{figure}[t]
  \includegraphics[width=\linewidth]{Figures/ToF_Cf252_Ej200SiPM.eps}
  \caption{Two-dimensional histogram of $^{252}$Cf time of flight measurements plotted against deposited energy in the EJ-200 stop detector. The inset show a logarithm projection onto the time of flight axis. The timing resolution was determined to be 544~ps from a gaussian fit to the gamma peak.}
  \label{fig:ToF_SiPM}
\end{figure}

Lower panels of Fig. \ref{fig:SiPMtiming}, show projections on the time axis of the histograms shown in the upper panel. The time resolution is obtained from  gaussian fits of the time distributions. The resolutions obtained for the non-amplified circuit and the amplified one are $\Delta t=469$~ps (FWHM), and $\Delta t=548$~ps (FWHM) respectively.

\subsection{Time of Flight with SiPMs}

As a proof of principle, a small scale time-of-flight setup (30~cm flight path) was implemented to measure neutron time-of-flight from a $^{252}$Cf source and determine the time-of-flight resolution. The setup consisted of a $20\times6\times6$~mm$^3$  piece of EJ-200 plastic scintillator attached to $6\times6$~mm$^2$ SensL\textsuperscript{\textregistered} SiPMs used as START detector and a  $100\times6\times6$~mm$^3$ scintillator bar attached to another pair of SensL\textsuperscript{\textregistered} SiPMs used as STOP. Both detectors were placed 30~cm apart and the same DAQ was used from the position resolution tests. The time-of-flight vs. deposited energy distribution can be seen in Figure \ref{fig:ToF_SiPM}, along with a 1-D projection showing the gaussian fit to the gamma-ray peak. Using the PolyCFD method, the time of flight resolution for the SiPM setup was determined to be $\Delta t=544$~ps (FWHM).
\begin{figure}[bt]
  \centering
  \begin{subfigure}{0.5\linewidth}
    \raggedleft
    \includegraphics[scale=0.2]{Figures/EJ276_SingleMylarBar_smallPMT_Sr90_LRvsEdep.eps}
  \end{subfigure}%
  \begin{subfigure}{0.5\linewidth}
    \raggedright
    \includegraphics[scale=0.18]{Figures/EJ276_SingleMylarBar_smallPMT_Sr90_LR.eps}
  \end{subfigure}%
  \caption{Left: Two-dimensional histogram of the time difference between PMTs on opposite ends of a 254~mm ESR wrapped bar of EJ-276. Right: Projection of 2D histogram on the time axis. A gaussian fit to the distribution is shown in red.}
  \label{fig:MylarTiming}
\end{figure}

The small scale SiPM timing tests establish SiPMs as viable detectors for small scale arrays. Testing will continue to determine scalability of SiPMs to a large, multi-detector resistive readout system.

\section{Eljen 276 Detector Tests}

 \subsection{Timing Tests with PMTs}
\begin{figure}[tp]
  \centering
 \begin{overpic}[scale=.35]{Figures/ToFvsEdep_Eljen276_Cf252.eps}
 \put(35,40){\includegraphics[scale=.18]{Figures/ToF_EJ276.eps}}
 \end{overpic}
 \caption{Two-dimensional histogram of $^{252}$Cf time of flight versus deposited energy in the mylar wrapped EJ-276 stop detector. The inset is a projection of the gamma-ray peak in the time of flight spectrum and has a 524~ps FWHM.}
 \label{fig:TOFEJ276}
\end{figure}
 
Timing performance of the $127\times12.7\times6$ mm\textsuperscript{3} EJ-276 bars was tested with fast, compact Hamamatsu R11265U photomultipliers. The bars of EJ-276 were machined from 12.7~mm thick sheets by Agile Technologies, Inc. and wrapped with 3M\textsuperscript{\texttrademark} ESR or Lumirror\textsuperscript{\texttrademark} (produced by Toray). Some bars were also provided with no wrapping to determine the effect of the reflective layers.
\begin{figure*}[htbp]
 \centering
  \includegraphics[width=0.9\textwidth]{Figures/PSDWrappingComparison.eps}
  \caption{Two dimensional histograms of the CCM PSD of three different types of wrapping.}
  \label{fig:PSDEJ276}
\end{figure*}
ESR is a specular reflector with 98\% reflectivity in the visible spectrum and Lumirror\textsuperscript{\texttrademark} is a diffuse reflector. Figure \ref{fig:MylarTiming} shows the 2D-histogram of left-right time difference against the deposited energy in the scintillator and the projection of the timing resolution results with the ESR wrapped EJ-276 bar. From a gaussian fit to the time difference projection the resolution is $\Delta t=543$~ps (FWHM). A thick mylar film on the $^{90}$Sr source caused the energy spectrum to be compressed by approximately 500~keV, as seen in the endpoint energy rougly around 1700~keV in Fig. \ref{fig:MylarTiming}.  The Lumirror\textsuperscript{\texttrademark} wrapped detector was not tested for timing due to the poor neutron-gamma discrimination that will be shown in Section \ref{PSDsection}.

A time-of-flight setup similar to the SiPM test was made to measure the time of flight resolution for a single 127~mm long bar of EJ-276 plastic scintillator wrapped with ESR. Utilizing the same start detctor from the SiPM time of flight setup and the EJ-276 bar coupled to PMTS as the stop detector, the $^{252}$Cf spectrum was measured again. From a fit to the gamma-ray peak in the the time of flight spectrum, the time resolution was determined to be 524~ps.

\subsection{n-${\mathit \gamma}$ discrimination} \label{PSDsection}
EJ-276 evolved from first-generation neutron-gamma discriminating plastic scintillator, EJ-299. Typical scintillators are cast in specific molds, limiting detector designs. EJ-276 is capable of being cut and polished in desired geometries to optimize detector light collection. The pulse shape discrimination response of EJ-276 is the same as EJ-299 with a slower neutron response decay than the gamma-ray response. The PSD response mechanism for EJ-299 can be found accurately described in \cite{Zaitseva2012}. The materials response to neutron and gamma-ray scattering was tested for the long narrow segments with different wrappings.
\subsubsection{Wrapping Tests}
 The neutron and gamma-ray responses from EJ-276 were recorded with the same 16 bit 250 MHz digitizer used for earlier setups, and the pulse shape discrimination was tested using the Charge Comparison Method \cite{CCMPSD}. This method was figured to be the most optimal when using a high bit-resolution digitizer \cite{HighResPSD}.
Two wrapped bars from Agile Technologies, Inc. (3M\textsuperscript{\texttrademark} ESR and Lumirror\textsuperscript{\texttrademark}) and a third bare bar wrapped with teflon were tested to measure the effect of the outer reflective layer on the pulse shape discrimination. Using the $^{252}$Cf source and a 2 inch block of lead to attenuate the large gamma-ray flux, the waveforms from each bar were digitized and tail to total integratin ratios were calculated. Figure \ref{fig:PSDEJ276} shows the PSD plots for bars wrapped with Lumirror\textsuperscript{\texttrademark} (a), teflon (b), and ESR (c) with figure of merits being calculated between 400 and 500~keVee. The figure of merits for the Lumirror\textsuperscript{\texttrademark}, teflon, and ESR bars were 0.820$\pm$0.012, 1.042$\pm$0.016, and 0.977$\pm$0.015 respectively. Future tests of the bar were only done with 3M\textsuperscript{\texttrademark} ESR wrapping to maintain the best neutron-gamma discrimination.

\section{Prototype} \label{Prototype}

\begin{figure}[ht]
  \begin{subfigure}{0.5\linewidth}
%    \raggedleft
  \includegraphics[width=\linewidth]{Figures/PrototypePositionMap.eps}
 \end{subfigure}%
  \begin{subfigure}{0.5\linewidth}
 %   \raggedright
  \includegraphics[width=\linewidth]{Figures/Prototype_Segmentation.jpg}
 \end{subfigure}%
  \caption{Shown is the position of scintillation within the segmented array, reconstructed using the position sensitive signals from the Vertilon Interface board. The source was located in the +X direction from the detector.}
  \label{fig:PSPMTImage}
\end{figure}

ESR wrapped EJ-276 bars were shown to meet NEXT design goals, facilitating the assembly of a 2x2 in$^{2}$ segmented detector array. The array has 4x8 segmentation, the higher segmentation being along the particle flight path (see Figure \ref{fig:PSPMTImage}). A full NEXT module is made up of one segmented array coupled to Hamamatsu H12700A position sensitive PMTs on each end of the array. The H12700A PSPMTs have an 8x8 segmentation (64 6x6 mm$^{2}$ anodes), with each anode having an individual readout. A Vertilon PSPMT Anger Logic interface board (Model SIB064B-1018) was used to reduce the position sensitive readout to 4 position signals. Figure \ref{fig:PSPMTImage} shows the reconstructed array segmentations from the Anger Logic position measurement of the PSPMTs \cite{ANGER}. The common dynode signal used for timing is passed through the interface board directly from the PSPMT. The segement dependent time-of-flight analysis calculates the neutron energies on a segment by segment basis (using the reconstructed positions).

\subsubsection{Time of Flight Measurements}
\begin{figure}[tp]
 \centering
 \includegraphics[width=0.9\linewidth]{Figures/CollimatedCo60_PrototypeToF.eps}
 \caption{NEXT prototype time of flight resolution using a collimated $^{60}$Co source at a flight distance of 439.7~mm. The gaussian fit to the distribution shows the time resolution (FWHM) is 548~ps. No position dependent correction or cut has been applied to the distribution. The time interval for a gamma-ray to cross the detector is negligible compared to the time resolution of the DAQ.}
 \label{fig:CollimatedCoToF}
\end{figure}
\begin{figure}[tb]
  \centering
  \includegraphics[width=0.9\linewidth]{Figures/Cf252_Spectrum_logY.eps}
  \caption{$^{252}$Cf neutron energy spectrum as measured with the NEXT prototype using the segment dependent analysis (blue). The red line shows the expected neutron yield based on a 100 keVee detection threshold.}
  \label{fig:Cf252Spectrum}
\end{figure}


\begin{figure}[t]
  \centering
 \begin{overpic}[scale=.35]{Figures/Prototype_PSD.eps}
 \put(35,35){\includegraphics[scale=.19]{Figures/Prototype_PSD_FOM.eps}}
 \end{overpic}
 \caption{Neutron-gamma discrimination from the common dynode signal of the PSPMT. The walk in the distribution has been removed by correcting the gamma-ray portion to a flat line on a bin by bin basis.}
 \label{fig:PSPMTPSD}
\end{figure}

To measure the full prototype time of flight resolution, a collimated $^{60}$Co source was used. Figure \ref{fig:CollimatedCoToF} shows the time of flight distribution for the entire prototype. From a gaussian fit to the distribution, the time of flight resolution is $\Delta$t=548~ps. Once the prototype was estalished to have high resolution timing, a proof-of-principle neutron energy measurement was made using the $^{252}$Cf source. The source was placed at a distance of 439.7~mm from the front face of the prototype. The neutron yield shown in Figure \ref{fig:Cf252Spectrum} was calculated using time-of-flight information and a neutron gate has been applied using the PSD information shown in Figure \ref{fig:PSPMTPSD}.  

\subsubsection{neutron-gamma discrimination}
The PSPMT response is different then that of the fast timing PMTs used to initially test EJ-276. This response affects the overall pulse shape, potentially affecting the pulse shape discrimination capabilities. The dynode signals contain the pulse shape discrimination information. The four position signals lose neutron-gamma information after passing through the resistive network. Figure \ref{fig:PSPMTPSD} shows the neutron-gamma discrimination using the CCM. Using the same energy cuts from the wrapping tests, the FOM is 1.070$\pm$0.016. The NEXT protype does not show any noticeable affect on neutron-gamma discrimination due to segmentation or multi-anode readout.


%%HERE the BIBLIOGRAPHY
%\section*{References}
%%\nocite{*}
%%\bibliography{mybibfile}
%\bibliography{NEXT_NIM}
%
%\end{document}
